\chapter{Test phase}

Using separate compilation, we have tested each algorithm separately, to ensure that they are all valid, taking into account the starting assumptions for each.

\section{Giant tour}

We choose the following parameters for the giant lathe algorithm :

\begin{itemize}
\item The first customer is customer 3 ($start = 3$).
\item The number of customers is 5 ($n=5$).
\item The distance matrix is as follows :

\begin{displaymath}
dist = \begin{bmatrix}
0.0	& 20.0	 & 25.0 	& 30.0 	& 40.0 	& 35.0 \\
20.0	& 0.0 	 & 10.0 	& 15.0 	& 40.0 	& 50.0 \\
25.0	& 10.0	 & 0.0 	& 30.0 	& 40.0 	& 35.0 \\
30.0	& 15.0	 & 30.0 	& 0.0 	& 25.0 	& 30.0 \\
40.0	& 40.0	 & 40.0 	& 25.0 	& 0.0 	& 15.0 \\
35.0	& 50.0	 & 35.0	& 30.0 	& 15.0 	& 0.0
\end{bmatrix}
\end{displaymath}
\end{itemize}

Applying the algorithm manually, we find a giant tour passing through customers $C_3$, $C_1$, $C_2$, $C_5$ and $C_4$ respectively.

This is consistent with the table constructed by the algorithm we implemented :

\begin{displaymath}
T = \begin{bmatrix}
3 & 1 & 2 & 5 & 4
\end{bmatrix}
\end{displaymath}

\section{Cutting procedure}

We choose the following parameters for the cutting procedure :

\begin{itemize}
\item The number of customers is 5 ($n=5$).
\item The array T corresponds to the array found with the giant lathe algorithm :

\begin{displaymath}
T = \begin{bmatrix}
3 & 1 & 2 & 5 & 4
\end{bmatrix}
\end{displaymath}

\item The vehicle capacity is 10 ($Q=10$).
\item The distance matrix is as follows :

\begin{displaymath}
dist = \begin{bmatrix}
0.0	& 20.0	 & 25.0 	& 30.0 	& 40.0 	& 35.0 \\
20.0	& 0.0 	 & 10.0 	& 15.0 	& 40.0 	& 50.0 \\
25.0	& 10.0	 & 0.0 	& 30.0 	& 40.0 	& 35.0 \\
30.0	& 15.0	 & 30.0 	& 0.0 	& 25.0 	& 30.0 \\
40.0	& 40.0	 & 40.0 	& 25.0 	& 0.0 	& 15.0 \\
35.0	& 50.0	 & 35.0	& 30.0 	& 15.0 	& 0.0
\end{bmatrix}
\end{displaymath}

\item Customers have the following requests :

	\begin{itemize}
		\item $C_1$ : 5
		\item $C_2$ : 4
		\item $C_3$ : 4
		\item $C_4$ : 2
		\item $C_5$ : 7
	\end{itemize}
\end{itemize}

The implemented algorithm gives the graph in the form of successor tables :

\begin{displaymath}
head = \begin{bmatrix}
0 & 2 & 4 & 5 & 7
\end{bmatrix}
\end{displaymath}

\begin{displaymath}
succ = \begin{bmatrix}
1 & 2 & 2 & 3 & 3 & 4 & 5 & 5
\end{bmatrix}
\end{displaymath}

\begin{displaymath}
cost = \begin{bmatrix}
60.0 & 65.0 & 40.0 & 55.0 & 50.0 & 70.0 & 90.0 & 80.0
\end{bmatrix}
\end{displaymath}

\section{Bellman algorithm}

Our implementation gives the following results :

\begin{displaymath}
potentiels = \begin{bmatrix}
0.0 & 60.0 & 65.0 & 115.0 & 185.0 & 205.0
\end{bmatrix}
\end{displaymath}

\begin{displaymath}
pere = \begin{bmatrix}
0 & 0 & 0 & 1 & 3 & 3
\end{bmatrix}
\end{displaymath}

The shortest path is through $H_0$, $H_1$, $H_3$ and $H_5$.

\section{Final results}

We then tested our program for each test file given by the subject. For the file {\textit cvrp\_100\_1\_det.dat} we obtain :

\begin{itemize}
\item Starting with customer 1 :

\begin{lstlisting}[language=bash, frame=shadowbox]
$ ./main cvrp_100_1_det.dat 1
Coût : 14289.270508
Tournée : 40 59 31
Tournée : 15 82 70 87 45 41 36
Tournée : 76 50 60 99 92
Tournée : 34 11 24 8 91 56 3
Tournée : 64 20 96 39 78
Tournée : 83 58 66 80 86 38
Tournée : 4 17 84 28 67
Tournée : 61 48 57 30 68 54 10
Tournée : 21 79 43 44 62 19 22 37
Tournée : 97 65 85 25 47 81
Tournée : 14 33 27 74 75 77 35 73 100
Tournée : 69 12 51 94 32 6
Tournée : 71 72 13 95 26 46 55 90 5
Tournée : 63 42 16 53 98
Tournée : 2 18 9 7 29 88
Tournée : 1 93 89 49 23 52
\end{lstlisting}

\item Starting with customer 3 :

\begin{lstlisting}[language=bash, frame=shadowbox]
$ ./main cvrp_100_1_det.dat 3
Coût : 14162.331055
Tournée : 10 4 67 83 58 34 11
Tournée : 87 24 8 91 45 41 36
Tournée : 31 76 50 60 99 92
Tournée : 15 86 38 64 40 59
Tournée : 61 48 43 54 68 70 82
Tournée : 44 62 19 22 37
Tournée : 7 9 18 79 21 81 80 57 30
Tournée : 14 20 96 39 78
Tournée : 12 51 94 32 6
Tournée : 97 2 52 46 55 90 5 69
Tournée : 47 33 27 74 75 77 35 73 100
Tournée : 66 17 84 28 65 85 25
Tournée : 23 49 89 93 1
Tournée : 53 98 71 72 13 95 26
Tournée : 3 56 29 88 63 42 16
\end{lstlisting}
\end{itemize}

For the file {\textit cvrp\_100\_1\_r.dat} we obtain :

\begin{itemize}

\item Starting with customer 1 :

\begin{lstlisting}[language=bash, frame=shadowbox]
$ ./main cvrp_100_1_r.dat 1
Coût : 1014.550049
Tournée : 26 12 76 50 78 34 35 71 66 65 29 24 4 41
Tournée : 2 57 15 43 38 86 13
Tournée : 93 85 91 100 37 98 61 16 44 14 42 87
Tournée : 89 6 94 95 97 92 59 99 96
Tournée : 62 10 90 32 63 64 46 8 45 17 84 5 60 83 18 52
Tournée : 31 88 7 82 48 47 36 49 19 11
Tournée : 80 68 77 3 79 33 81 9 51 20 30 70
Tournée : 73 72 74 22 75 56 39 23 67 25 55 54
Tournée : 1 69 27 28 53 58 40 21
\end{lstlisting}

\item Starting with customer 59 :

\begin{lstlisting}[language=bash, frame=shadowbox]
$ ./main cvrp_100_1_r.dat 59
Coût : 992.699951
Tournée : 26 4 24 29 78 34 35 71 66 65 52
Tournée : 75 56 39 23 67 25 55 54 80 68 12
Tournée : 87 2 57 15 43 38 41 22 74 72 73 21
Tournée : 62 10 90 32 63 64 46 8 45 17 86 44 14 42
Tournée : 31 88 7 82 48 47 36 49 19 11
Tournée : 69 1 50 76 77 3 79 33 81 9 51 20 30 70
Tournée : 53 28 27
Tournée : 18 60 83 84 5 61 16 91 100 97 95 13 58 40
Tournée : 59 92 37 98 85 93 99 96 94 6 89
\end{lstlisting}
\end{itemize}

For the file {\textit cvrp\_100\_1\_c.dat} we obtain :

\begin{itemize}
\item Starting with customer 1 :

\begin{lstlisting}[language=bash, frame=shadowbox]
$ ./main cvrp_100_1_c.dat 1
Coût : 882.290039
Tournée : 55 57 59 54 53 56 58 60
Tournée : 81 78 76 71 70 73 77 79 80
Tournée : 99 96 95 94 93 92 97 100 98
Tournée : 18 17 13 15 16 14 12 19
Tournée : 34 36 39 38 37 35 31 32 33
Tournée : 20 21 22 23 26 28 27 25 24 29 30
Tournée : 40 41 42 44 45 46 48 50 51 52 49 47 43
Tournée : 63 65 67 66 69 62 74 72 61 64 68
Tournée : 91 89 88 85 84 83 82 86 87 90
Tournée : 1 2 4 3 5 7 8 9 6 11 10 75
\end{lstlisting}

\item Starting with customer 13 :

\begin{lstlisting}[language=bash, frame=shadowbox]
$ ./main cvrp_100_1_c.dat 13
Coût : 856.509949
Tournée : 80 79 77 73 70 71 76 78 81
Tournée : 59 57 55 54 53 56 58 60
Tournée : 34 36 39 38 37 35 31 32 33
Tournée : 20 21 22 23 26 28 27 25 24 29 30
Tournée : 40 41 42 44 45 46 48 50 51 52 49 47 43
Tournée : 63 65 67 66 69 62 74 72 61 64 68
Tournée : 91 89 88 85 84 83 82 86 87 90
Tournée : 98 96 95 94 93 92 97 100 99
Tournée : 10 8 9 6 4 3 5 7 75 1 2
Tournée : 13 17 18 19 15 16 14 12 11
\end{lstlisting}
\end{itemize}

All the costs obtained coincide with the expected results given in the subject.

\subsection{Memory leaks}

We also checked that our program didn't cause memory leaks by using the utility \textit{valgrind}.

\begin{lstlisting}[language=bash, frame=shadowbox, breaklines=true]
==15006== 
==15006== HEAP SUMMARY:
==15006==     in use at exit: 0 bytes in 0 blocks
==15006==   total heap usage: 33 allocs, 33 frees, 1,050,816 bytes allocated
==15006== 
==15006== All heap blocks were freed -- no leaks are possible
\end{lstlisting}
